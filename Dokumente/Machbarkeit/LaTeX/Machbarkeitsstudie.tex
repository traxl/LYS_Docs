\documentclass[a4paper,12pt]{scrreprt}
\usepackage[T1]{fontenc}
\usepackage[utf8]{inputenc}
\usepackage[ngerman]{babel}
\usepackage[table]{xcolor}% http://ctan.org/pkg/xcolor
%\usepackage{\textit{graphicx\textit{\textit{}}}} <- ??
\usepackage{graphicx}



\begin{document}


\author{Dominik Backhausen\and Daniel Dimitrijevic\and Alexander Rieppel\and Thomas Traxler}
\title{Machbarkeitsstudie\\ LAN Yourself}
\date{09.10.2013}
\maketitle
\tableofcontents



\chapter{Projekt-Team}
	
	Name:Dominik Backhausen\\
	    F"ahigkeiten:Java, Html, C/C++, SQL\\\\
	  	Name:Daniel Dimitrijevic
	\\ 	F"ahigkeiten: Java, Html,C/C++, SQL    
	\\
	\\  Name:Alexander Rieppel
	\\
	    F"ahigkeiten:Java, Html, C/C++, SQL
	    \\
	    \\
	    Name:Thomas Traxler
	    \\
	    F"ahigkeiten:Java, XML, C/C++, SQL
	    \\
\chapter{Projektbeschreibung}
Ziel dieses Projektes ist es eine einfach zu bedienende, performante und m"oglichst gut kompatible VPN-Software zu entwickeln welche auf Peer-to-Peer Technologie basiert.\\

Es existieren auf dem Markt zwar bereits ziemlich solide und ausgereifte Softwarel"osungen auf diesem Gebiet, jedoch sind diese was den Funktionsumfang betrifft oft nicht zufriedenstellend. Dieses Produkt beinhaltet zwar ebenfalls keine grundlegend neuen Technologien und Funktionen, allerdings wird die von uns vorgesehene Kombination an Funktionsumfang und Bedienbarkeit nach ermessen des Projektteams aktuell von niemandem Angeboten. Wie bereits erw"ahnt wird besonderer Wert auf Performance, Kompatibilit"at und Bedienbarkeit gelegt, deshalb eignet sich die Software besonders gut f"ur Benutzer, die "uber wenig Know-How verf"ugen und ein schnell einzurichtendes und performantes VPN-Netzwerk ben"otigen. Jedoch sind die Einsatzm"oglichkeiten sehr breit gef"achert, was auch einen Einsatz als Firmen-VPN-Software nicht ausschlie"st oder Benutzer die spezielle Einstellungen f"ur den Einsatz ihres Programms innerhalb des VPN-Netzwerks ben"otigen. Zu diesem Anlass wird es ebenfalls die M"oglichkeit geben, die Konfigurationsdateien manuell zu bearbeiten um erweiterte Funktionen des VPNs zu nutzen.\\


\chapter{Voruntersuchung des Produkts}
	\section{Ist-Erhebung}
	
	Als VPN\footnote{Virtual Privacy Network} wird ein verschl"usseltes privates Netzwerk bezeichnet, dass prinzipiell "uber einen eigenen Server aufgebaut und von diesem verwaltet wird. Eine weitere M"oglichkeit stellt die Direktverbindung von Benutzer zu Benutzer dar welche als Peer-to-Peer bezeichnet wird. Bei diesem Verfahren gibt es keinen Server der die Verbindungen verwaltet, sondern dieses Management "ubernimmt jeder Benutzer f"ur sich selbst. Allerdings ist es auch verbreitet sich auf bestimmte Einstellungen, wie bspw. IP-Adressbereich, pauschal zu einigen. Ebenfalls existiert die M"oglichkeit ein VPN hybrid aufzubauen, sodass zwar ein Server am Verbindungsaufbau beteiligt ist, allerdings bei der Verbindung selbst keine Rolle mehr spielt. Auf dieses virtuelle Netzwerk haben ausschlie"slich die Teilnehmer des selben Zugriff und es verspricht dar"uber hinaus eine hohe Sicherheit durch Verschl"usselung, die bis zu einem gewissen Grad auch vor R"uckverfolgung sch"utzen kann. Allerdings kann ein VPN auch dazu genutzt werden gewisse Barrieren zu umgehen, um z.B. eine Verbindung mit dem Intranet einer Firma herzustellen und dort auf einen lokalen Server zuzugreifen, was jedoch ebenso positive wie negative Aspekte mit sich bringt welche f"ur jeden Fall gesondert eingesch"atzt werden m"ussen. Im Folgenden ist eine Marktanalyse ausgef"uhrt um existente Dienste und ihre Schw"achen aufzuzeigen:
	
	\subsection{Hamachi} ist ein gehosteter VPN-Service, "uber welchen Benutzer durch virtuelle Netzwerke miteinander verbunden werden k"onnen. Es erlaubt mehrere solcher Netzwerke zu erstellen und zu verwalten und basiert prinzipiell zu einem gro"sen Teil auf Peer-to-Peer. Erst wenn keine direkte Verbindung hergestellt werden kann wird der Client "uber einen Server mit dem restlichen Netzwerk verbunden. Ein gro"ser Nachteil hierbei ist, dass durch die Verbindung mit einen Server die Kompatibilit"at mit anderen Programmen leidet. Der Grund daf"ur ist, dass Hamachi offenbar keine Ports zu den entsprechenden Nutzern weiterleitet und somit andersartige Dienste die nicht Port 80 verwenden schlicht ignoriert werden. Dar"uber hinaus ist der Benutzer hiermit auch an die Leistungsf"ahigkeit und Verbindungsgeschwindigkeit der Server gebunden, den er sich im Normalfall mit anderen Benutzern teilen muss. Dies hat unter Umst"anden zur Folge, dass die Verbindung unter hoher Serverlast sehr langsam wird oder nicht mehr funktioniert. Ein weiterer gro"ser Nachteil des Dienstes ist, dass in der kostenlosen Version  nur eine begrenzte Anzahl von Benutzern in einem Netzwerk erlaubt ist. Da diese Obergrenze sehr gering gehalten wurde, wird ein vern"unftiges Arbeiten im Netzwerk teilweise nicht mehr m"oglich. Abgesehen davon ist die Benutzbarkeit des GUI zwar vergleichsweise gut, jedoch ist die allgemeine Geschwindigkeit des Programms selbst eher schlecht zu bewerten. 
	
	\subsection{OpenVPN GUI} bietet eine grafische Oberfl"ache zur bekannten VPN-L"osung OpenVPN. Sie ist minimalistisch gehalten und bietet nur wenige Einstellungen, wie Proxy und Sprache. Alle anderen Einstellungen sind ausschlie"slich, in dem f"ur die Verbindung vorgesehenen OpenVPN Config-File zu t"atigen. Bereits hier besteht die M"oglichkeit, dass sich unerfahrene Benutzer  "uberfordert sehen. Zwar existieren im Netz diverse Config-File Editoren um die Files zu erstellen und zu editieren, allerdings sind auch diese nicht immer leicht zu handhaben. Dar"uber hinaus ist ein eventuelles Troubleshooting, dass "uber eine triviale Neuinstallation hinaus geht, f"ur einen unerfahrenen Benutzer einfach nicht zumutbar und erfordert auch von erfahreneren Benutzern oft mehrere Stunden Aufwand um ein Problem zu l"osen. Auch die Meldungen in der Konsole der Anwendung sind f"ur einen Laien alles andere als verst"andlich, was ein eventuelles Troubleshooting komplizierter macht. Abgesehen davon basiert OpenVPN in erster Linie auf einer Client-Server Architektur, wozu auch noch die Tatsache kommt, dass ein OpenVPN Server nicht gerade leicht zu installieren ist.
	
	
	\subsection{ShellfireVPN} ist ein VPN-Dienst der in der kostenlosen Version, ausschlie"slich eine Unterst"utzung f"ur Windows bietet und in seiner Funktionalit"at sehr eingeschr"ankt ist. Auch die Bandbreite und Sicherheit ist in der kostenlosen Version stark begrenzt. Dar"uber hinaus basiert ShellfireVPN auf OpenVPN und damit auf einer Server-Client Architektur. Wenn mehr Bandbreite gefordert ist, wird der Dienst kostenpflichtig. Auch hat der Benutzer in der kostenlosen Version von ShellfireVPN ausschlie"slich Zugang zu Servern in Deutschland. Der einzige Punkt wo der Dienst gemessen an anderen L"osungen "uberdurchschnittlich abschneidet ist die Benutzerfreundlichkeit.
	
	\subsection{CyberGhost VPN} ist ein weiterer Dienst der auf OpenVPN aufsetzt und in der kostenlosen Version eingeschr"ankte Funktionalit"at und begrenzte Bandbreite bietet. Benutzerfreundlichkeit ist ebenfalls gemessen an anderen L"osungen "uberdurchschnittlich und auch ist der Vorgang der im Hintergrund abl"auft speziell f"ur den Laien abstrahiert. Auch hier fallen zus"atzliche Kosten f"ur mehr Bandbreite an und man ist an die Verf"ugbarkeit der Server in den entsprechenden L"andern gebunden. 
	
		
	\section{Soll-Zustand}
		
		\subsection{Allgemeine Beschreibung}
		
		Im Rahmen des Projektes soll eine leicht bedienbare Software entwickelt werden, die den Aufbau eines virtuellen lokalen Netzwerkes sicherstellt. Wichtig ist dabei, dass diese Software nicht auf einer Server-Client Architektur basiert, sondern als Peer-to-Peer L"osung realisiert wird. Aus diesem Grund ist die Software an keine l"anderspezifischen Grenzen gebunden und an sich global verf"ugbar. Allerdings ist eine hybride L"osung ebenfalls nicht ausgeschlossen. Dies bedeutet, dass ein Server zwar am eigentlichen Verbindungsaufbau in geringem Ma"se (z.B. Client Lokalisation) im Einsatz ist, allerdings an der Verbindung selbst nicht mehr beteiligt ist. Dabei ist zu beachten, dass es kein Ziel ist die VPN-L"osung selbst zu entwickeln, sondern lediglich die Verwaltungssoftware, eine zugeh"orige Benutzeroberfl"ache und die Funktionalit"aten die auf eine bestehende VPN-L"osung aufgesetzt werden.
		
		Zu den wichtigsten Punkten geh"oren gute Kompatibilit"at und Performance die grunds"atzlich zu jeder Zeit von der Software gew"ahrleistet werden m"ussen. Um dies sicherzustellen kann die Konfiguration der eigentlichen VPN-Basis zu jeder Zeit "uber das geplante Programm ge"andert werden. Da die Software auch f"ur Benutzer mit wenig Erfahrung gedacht ist, wird sehr viel wert auf eine intuitive und m"oglichst dezente Benutzeroberfl"ache gelegt.
		
			
		\subsection{Muss-Ziele}
		\begin{itemize}
		\item Peer-to-Peer Prinzip (gegebenenfalls Hybrid)
		\item Erstellung von VANs
		\item intuitives GUI-Design
		\item Konfigurationsdateien f"ur erweiterte Einstellungen
		\item Netzwerkteilnehmerliste
		\item Senden von Textnachrichten
		\end{itemize}
		
			
		\subsection{Kann-Ziele}
			\begin{itemize}
			\item Weiterleitung der Internetverbindung
			\item Einstellung ob die Internetverbindung des LAN-Netzwerks oder die eigene verwendet werden soll
			\item Zugriffspunkt f"ur die Internetverbindung im VAN nach bestimmten Kriterien aussuchen lassen, wobei dies entweder manuell oder "uber einen Algorithmus erfolgen kann
			
			\item Ports am Internetzugriffspunkt zu bestimmten Netzwerkteilnehmern weiterleiten
			\item Gemeinsame Ordner erstellen und bestimmten Netzwerkteilnehmern freigeben
			
			\item Onion-Routing (Pfadauswahl auch durch Kriterien m"oglich)
			
			\end{itemize}
			
		\subsection{Nicht-Ziele}
			Es ist kein Ziel die Software nach ihrer Fertigstellung zu vermarkten. Des weiteren ist es ebenfalls kein Ziel auf einer eigens programmierten VPN-Basis aufzusetzen. Die Software wird ausschlie"slich auf Open-Source L"osungen zur"uckgreifen und in ihrer Grundform kostenlos zur Verf"ugung stehen. Des Weiteren ist es zwar m"oglich, dass die Software nach der Fertigstellung weiterhin durch das Projektteam betreut wird, allerdings ist diese eventuelle Weiterf"uhrung nicht Teil dieses Projektes.
			
	\section{Festlegen der Hauptfunktionen}
		\textbf{/HF10/ Verbinden mit einem VAN bzw. VAN erstellen}
		
		Jeder Benutzer muss sich mittels eines Benutzernamens erkennbar machen, Identifikation erfolgt jedoch mittels anderer Parameter (IP und Host-File). \\
		
		
		\textbf {/HF20/ Netzwerkteilnehmer anzeigen
		}\\
		Umfasst die Anzeige von:
		
		\begin{itemize}
		\item Latenzzeit
		\item IP-Adresse
		\item Benutzername
		\item Zugriffspunkt (ja/nein) \footnote{Kannziel}\\
		\end{itemize}
		
		
		
		
		
		\textbf {/HF30/ Textnachrichten an Netzwerkteilnehmer senden}
		\\Auch eine Kommunikation "uber Textnachrichten unter den Teilnehmern wird m"oglich sein.
		
		 \textbf {/HF40/ Konfigurationsdateien f"ur erweiterte Einstellungen}
		\\Dadurch wird erfahrenen Benutzern die M"oglichkeit gegeben erweiterte Einstellungen, die in der VPN-Basis verf"ugbar sind, zu verwenden.
		
	\section{Festlegung der Nebenfunktionen}
	\textbf { /NF10/ Internetverbindung im VAN freigeben} 
	\\Ein Teil oder der gesamte Internettraffic (falls nur ein Teilnehmer seine Internetverbindung freigegeben hat), den andere Teilnehmer produzieren, wird "uber diese Verbindung geroutet.
			
	\textbf {/NF20/ Mit dem Internet verbinden} 
	\\Der Benutzer kann festlegen ob er standardm"a"sig seine eigene Internetverbindung oder "uber einen Zugriffspunkt\footnote{Der Zugriffspunkt ist hier ein User der zur Verbindung mit dem Internet dient, d.h. seine Internetverbindung freigegeben hat} in seinem VAN ins Internet gehen m"ochte.
	
	\textbf {/NF30/ Netzwerkteilnehmer f"ur Internetverbindung nach Kriterien ausw"ahlen lassen}
	
	\textbf {/NF31/ Netzwerkteilnehmer f"ur Internetverbindung manuell ausw"ahlen}
	\\Es soll m"oglich sein einen bestimmten Netzwerkteilnehmer, aus dem Pool der Teilnehmer die ihre Verbindung freigegeben haben auszuw"ahlen, um "uber diesen Teilnehmer eine Internetverbindung aufzubauen.
	
	\textbf {/NF32/ Netzwerkteilnehmer f"ur Internetverbindung nach Algorithmus bestimmen}
	\\Ebenfalls soll es m"oglich sein den optimalen Netzwerkteilnehmer automatisch nach einem bestimmten Kriterium (Latenzzeit, zur Verf"ugung stehende Bandbreite, etc.) ausw"ahlen zu lassen.
	
	\textbf {/NF40/ Gemeinsamen Ordner erstellen}
	\\Eine weitere Funktion ist das Erstellen eines gemeinsamen Ordners, der innerhalb des Netzwerks freigegeben werden kann. Zus"atzlich wird es m"oglich sein Berechtigungen f"ur diesen Ordner festzulegen und die Benutzer auszuw"ahlen, die auf ihn zugreifen d"urfen.
	
	\textbf {/NF50/ Netzwerkteilnehmer ignorieren}
	\\Ein Netzwerkteilnehmer kann andere Netzwerkteilnehmer von der Software ignorieren lassen. Dies bedeutet, dass die Kommunikation "uber das Netzwerk zwischen diesen Teilnehmern nicht l"anger m"oglich ist solange diese Sperre besteht.
	
	\section{Festlegen der Hauptdaten}
	\textbf {/HD10/ Benutzername}
	
	Verwendeter Benutzername\\
	\textbf {/HD20/ Gespeicherte Daten der anderen Netzwerkteilnehmer}
	
	Daten wie IP-Adresse oder Ports die f"ur die Verbindung verwendet werden sollen\\
	\textbf {/HD30/ Einstellungen}
	
	Vorgenommene Einstellungen in der Software\\
	
	
	\section{Festlegen der Hauptleistungen}
	
	   \textbf{/LL10/ Optimierter Datenverkehr durch Peer-to-Peer}
		
		Da hier keine Server-Client Architektur verwendet wird, ist auch der Datenverkehr, der "uber einen einzigen Netzwerkteilnehmer gef"uhrt wird, deutlich geringer.\\
		\textbf {/LL20/ Hohe Erweiterbarkeit der Software}
		
		Das Endprodukt soll die Bedingungen der einfachen Wartbarkeit und Erweiterbarkeit erf"ullen. Dies bedeutet das Softwarecode sowohl verst"andlich geschrieben als auch gut dokumentiert ist, au"serdem werden die Programmteile einer klaren Aufteilung unterliegen.\\
		\textbf {/LL30/ Hohe Kompatibilit"at zu verschiedenen Programmen}
		
		Bei der Entwicklung der Software wird besonders darauf geachtet, dass sie mit verschiedenen Programmen kompatibel ist.\\
		\textbf {/LL40/ Hohe Performanz}
		
		Hiermit ist sowohl Performanz innerhalb des Programms, als auch im Netzwerk gemeint.
		
	\section{Festlegen der wichtigsten Aspekte der Benutzerschnittstelle}
		
	Bestehende Programme haben oft eine unzureichend intuitive oder nicht ausreichend erkl"arte Benutzeroberfl"ache, in denen sich unerfahrene Benutzer nur sehr schwer zurechtfinden. Deshalb wird das Programm besonders intuitiv und simpel verwendbar sein. Es wird dar"uber hinaus ein kleines Hilfemen"u geben um Einsteigern die Benutzung noch einfacher zu gestalten.
	
	
		
	\section{Festlegen der wichtigsten Qualit\"atsmerkmale}
		
	\subsection{Sicherheit}
	Das Produkt wird stark auf den Einsatz durch viele Benutzer zugeschnitten. Um in dieser Gemeinschaft dem Abh"oren von "ubertragenen Daten durch Au"senstehende vorzubeugen, werden alle Daten verschl"usselt.
		 
	\subsection{Benutzbarkeit}
	Bei der Entwicklung wird "au"serst viel Wert auf eine intuitive und einfache GUI gelegt, die sowohl die Performanz als auch das Design betreffend, die erw"unschten Ergebnisse liefert und damit dem Benutzer zugute kommt, anstatt lediglich ein notwendiges "Ubel darzustellen. 
			
	\subsection{Effizienz}
	Da bei unserem Programm  mit vielen Benutzern gerechnet wird und diese auch viele Daten verschicken, muss unser Programm imstande sein eine durchgehende und stabile Verbindung zu unterst"utzen. Dies bedeutet in erster Linie die notwendigen Neukonfigurationen an der VPN-Basis zu optimieren. Weiters ist es notwendig, dass die Software zeitkritische Aktionen priorisiert und daher schnell und effizient abarbeitet.
		
		
	\subsection{Erweiterbarkeit}
	Wie bereits bei den Hauptleistungen erw"ahnt, ist vorgesehen das Programm so zu schreiben, dass zuk"unftige Erweiterungen der Software leicht m"oglich sind. 
		
	\subsection{Übertragbarkeit}
	Da entschieden wurde das Programm f"ur Windows zu optimieren, wird in weiterer Hinsicht weniger Wert auf die Übertragbarkeit der Software zu anderen Betriebssystemen gelegt. Dennoch wird es m"oglich sein (durch Zutun des Benutzers) eigene Versionen zu erstellen, die in weiterer Folge auch auf Linux oder Mac OSX lauff"ahig sind.
			
			
\chapter{Durchf"uhrbarkeitsuntersuchung}
	
	\section{Pr"ufen alternativer L"osungsvorschl"age}
	
		\subsection{Programmiersprachen}
		
		Prinzipiell ist es m"oglich dieses Projekt in allerlei Programmiersprachen umzusetzen. Da das Projektteam allerdings "uber einige Erfahrung in der Programmiersprache Java verf"ugt und auch haupts"achlich mit dieser Sprache arbeitet, wird dieses Projekt in Java entwickelt. Andere Sprachen f"ur die Entwicklung der Softwarel"osung zu verwenden w"urde zus"atzlichen Aufwand bedeuten, der keine ausreichenden Vorteile bieten kann.
		
	\section{Pr"ufen der technischen Durchf"uhrbarkeit}
		\subsection{Softwaretechnische Durchf"uhrbarkeit}
		
		Jegliche Software die f"ur einen einwandfreien Betrieb notwendig ist, wird bereits innerhalb der Auslieferungsversion mitgeliefert. Daher stellt die softwaretechnische Durchf"uhrbarkeit kein Risiko dar.
		
		\subsection{Hardwaretechnische Durchf"uhrbarkeit}
			
		Damit diese Software lauff"ahig ist, wird ein PC mit zumindest Microsoft Windows XP und einer funktionierenden Netzwerkkarte, die eine Verbindung mit dem Internet herstellen kann, ben"otigt. Da dies so gut wie jeder PC besitzt und auch zumindest Microsoft Windows XP auf den meisten PCs l"auft, ist die hardwaretechnische Durchf"uhrbarkeit weitgehend gegeben.
		
			
		\subsection{Verf"ugbarkeit von Entwicklungs- und Zielmaschinen}
			
	Die Software wird auf PCs mit Microsoft Windows 7 entwickelt. Da jedes Projektteammitglied "uber mehr als einen PC verf"ugt auf dem ein Microsoft Windows Betriebssystem lauff"ahig ist, kann die Verf"ugbarkeit von Entwicklungsmaschinen auf jeden Fall als gegeben betrachtet werden.\\\\Als Zielmaschinen sind alle PCs mit Microsoft Windows ab Windows XP bis Windows 8.1 gedacht. Die Portierung der Software auf andere Betriebssysteme, sollte zwar prinzipiell kein Hindernis darstellen, jedoch werden Versionen dieser Software, die f"ur andere Betriebssysteme als Windows gedacht sind nicht offiziell bereitgestellt. Trotzdem kann die Verf"ugbarkeit von Zielmaschinen als unbedenklich betrachtet werden, da Microsoft Windows noch immer das am weitesten verbreitete Betriebssystem ist.
			
	\section{Pr"ufen der personellen Durchf"uhrbarkeit}
		
		\subsection{Qualifikation der Fachkr"afte}
		
	Das gesamte Projektteam kann bereits mit einiger netzwerktechnischen Erfahrung aufwarten und hat sich ebenfalls bereits mit VPN-Software verschiedenster Art in der Vergangenheit besch"aftigt. Dar"uber hinaus verf"ugt das Team "uber ein hohes Ma"s an softwaretechnischem Wissen und hat bereits Erfahrung mit einigen Programmiersprachen. Da die Software zur G"anze in Java geschrieben wird und das Team schon einige Erfahrung mit dieser Programmiersprache gemacht hat, sieht es sich imstande das Projekt zu einem erfolgreichen und zufriedenstellenden Ende f"uhren.
	
		
		\subsection{Zusammenarbeit der Teammitglieder}
		
		
	Da die Teammitglieder schon "ofters in dieser Gruppenzusammensetzung gearbeitet haben und es sich daher als eingespieltes Team bezeichnen kann, geht es davon aus, dass dies auch in diesem Projekt zu einem positiven Abschluss beitr"agt.
	
	Weiters wurden bereits im Vorfeld m"oglichst effiziente Organisationsstrukturen und Kommunikationsplattformen geschaffen, welche der Effizienz der Projektarbeit eine Steigerung verschaffen sollten.
	
			
	\section{Risikoanalyse}
		
		\subsection{Personelle Risiken}
	Da dieses Projekt von einem Team, bestehend aus 4 Mitgliedern, bearbeitet wird und der schlimmste Fall, das mehrere Teammitglieder "uber einen gr"o"seren Zeitraum gleichzeitig ausfallen, eher eine  Ausnahme darstellt, sehen wir keine gr"o"seren personellen Risiken an diesem Projekt.
	
		
		\subsection{Technische Risiken}
	Neben technischen Risiken wie zu hoch gesteckte Ziele und zu geringes technisches Know-How, welche als eher gering eingestuft werden k"onnen, ist das Risiko, bez"uglich des  verwendeten VPN-Treibers, umso h"oher. Auf diesen hat das Projektteam in Funktionsweise und -umfang kaum bis keinen Einfluss und muss sich voll und ganz auf ihn verlassen k"onnen. M"ogliche Risiken die es beim verwendeten VPN-Treiber zu ber"ucksichtigen gilt, sind fehlender Funktionsumfang, nicht vorhergesehene Funktionsweisen oder zu hoher Aufwand um bestimmte Projektziele, mit dem Treiber als Grundlage, zu erreichen. Deshalb ist die Auswahl eines geeigneten VPN-Treibers besonders essenziell, um das Risiko dahingehend m"oglichst zu minimieren. Zus"atzlich beh"alt sich das Projektteam das Recht vor, diesen VPN-Treiber jederzeit zu wechseln falls im Laufe des Projektes wesentliche M"angel am Treiber festgestellt werden und damit nicht alle Muss-Ziele erf"ullt werden k"onnen.
		
			
\chapter{Nutzenanalyse}
	
	\section{Nutzen f"ur den Kunden}
	Die Software wird sich von anderen L"osungen auf diesem Gebiet vor allem dadurch abheben, dass sie eine h"ohere Programmkompatibilit"at bietet und sehr einfach zu handhaben ist. Zus"atzlich kann die Software mit einer hohen Effizienz und Performanz aufwarten.
	
	\section{Nutzen f"ur das Projektteam}
	
	Der Nutzen f"ur das Team besteht in erster Linie in der Aneignung neuer Erfahrungen und F"ahigkeiten in den Bereichen der Teamarbeit, Projektarbeit sowie Java-Programmierung. Ebenso wird das Team durch die genaue Auseinandersetzung mit VPN-Software und Treibern auch dahingehend einige Informationen und Erfahrungen sammeln.\\
	Des weiteren stellt das resultierende Produkt eine Basis dar, auf welche in Zuk"unftigen Projekten leicht aufgebaut werden kann, bzw. auch leicht erweitert werden kann. Denkbar w"aren hier zum Beispiel erweiterte Sicherheitsmechanismen f"ur Firmen.
	

\end{document}